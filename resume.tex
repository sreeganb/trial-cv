%%%%%%%%%%%%%%%%%%%%%%%%%%%%%%%%%%%%%%%%%
% "ModernCV" CV and Cover Letter
% LaTeX Template
% Version 1.1 (9/12/12)
%
% This template has been downloaded from:
% http://www.LaTeXTemplates.com
%
% Original author:
% Xavier Danaux (xdanaux@gmail.com)
%
% License:
% CC BY-NC-SA 3.0 (http://creativecommons.org/licenses/by-nc-sa/3.0/)
%
% Important note:
% This template requires the moderncv.cls and .sty files to be in the same 
% directory as this .tex file. These files provide the resume style and themes 
% used for structuring the document.
%
%%%%%%%%%%%%%%%%%%%%%%%%%%%%%%%%%%%%%%%%%

%----------------------------------------------------------------------------------------
%	PACKAGES AND OTHER DOCUMENT CONFIGURATIONS
%----------------------------------------------------------------------------------------

\documentclass[11pt,a4paper,roman]{moderncv} % Font sizes: 10, 11, or 12; paper sizes: a4paper, letterpaper, a5paper, legalpaper, executivepaper or landscape; font families: sans or roman

\moderncvstyle{classic} % CV theme - options include: 'casual' (default), 'classic', 'oldstyle' and 'banking'
\moderncvcolor{blue} % CV color - options include: 'blue' (default), 'orange', 'green', 'red', 'purple', 'grey' and 'black'

\usepackage{lipsum} % Used for inserting dummy 'Lorem ipsum' text into the template
\usepackage{graphicx}
%\usepackage{hyperref}
\usepackage[scale=0.815]{geometry} % Reduce document margins
%\setlength{\hintscolumnwidth}{3cm} % Uncomment to change the width of the dates column
%\setlength{\makecvtitlenamewidth}{10cm} % For the 'classic' style, uncomment to adjust the width of the space allocated to your name

%----------------------------------------------------------------------------------------
%	NAME AND CONTACT INFORMATION SECTION
%----------------------------------------------------------------------------------------

\firstname{Sree ganesh} % Your first name
\familyname{Balasubramani} % Your last name

% All information in this block is optional, comment out any lines you don't need
\title{Resume}
%\address{IISER-TVM}{Thiruvananthapuram}
\mobile{949-302-3736}
%\phone{}
%\fax{(000) 111 1113}
\email{sreeganb@uci.edu}
%\social[github]{sreeganb}
%\homepage{staff.org.edu/~jsmith}{staff.org.edu/$\sim$jsmith} % The first argument is the url for the clickable link, the second argument is the url displayed in the template - this allows special characters to be displayed such as the tilde in this example
%\extrainfo{$5^{th}$ year Int. BS-MS}
%\photo[70pt][0.4pt]{pictures/picture} % The first bracket is the picture height, the second is the thickness of the frame around the picture (0pt for no frame)
%\quote{"A witty and playful quotation" - John Smith}

%----------------------------------------------------------------------------------------

\begin{document}

\makecvtitle % Print the CV title

%----------------------------------------------------------------------------------------
%	EDUCATION SECTION
%----------------------------------------------------------------------------------------
\section{Summary}
\cvitem{}{Postdoctoral researcher with 6 years experience developing the quantum chemistry program package TURBOMOLE 
and 2 years experience in application of molecular dynamics methods (using CHARMM, Python) for studying the mechanisms and 
free energies of enzyme catalysis reactions and developing inhibitors.}

\section{Education}
\cventry{2020--present}{Post doctoral research associate}{University of Arizona}{Tucson}{}{}  % Arguments not required can be left empty
\cventry{2014--2020}{Doctor of Philosophy (PhD) in Chemistry}{University of California}{Irvine}{\textit{CGPA -- 3.9/4}}{}  % Arguments not required can be left empty
\cventry{2009--2014}{Dual degree BS-MS in Chemistry}{Indian Institute of Science Education and Research (IISER)}{Trivandrum (TVM)}{\textit{CGPA -- 8.3/10}}{}
%
\section{Relevant research experiences}
\cventry{2020--present}{The University of Arizona}{}{}{}{}
\cvitem{}{Postdoctoral research associate in Prof. Schwartz group}
%\cvitem{3}{Algorithms for quantum chemical calculations.}
%\cvitem{}{}
%\subsection{Broad Research Area: Theoretical chemistry}
\cvitem{1}{Developed and implemented molecular dynamics (MD) based method for the calculation 
of free energies of enzyme catalysis reactions. Used Python and shell scripting to develop workflow for 
analysis of MD trajectories which were obtained using the CHARMM software.}
\cvitem{2}{Used the free energy method that I implemented to study the human Mat2a enzyme catalysis reaction. Extensively used 
CHARMM, VMD and python for the analysis of MD trajectories and study the active site cavity to further develop inhibitors for this 
enzymatic reaction.}
\cvitem{3}{Applied the transition path enhanced sampling and free energy methods to study the malarial plasmodium vivax adenosine deaminase 
enzyme catalysis reaction mechanism.}
\cvitem{4}{Mentored graduate students through projects involving QM/MM simulations.}
\cventry{2014--2020}{University of California, Irvine}{}{}{}{}
\cvitem{}{Graduate student researcher in Prof. Furche group}
\cvitem{1}{Developed the theory and implemented second order properties within the random phase approximation density functional method 
into the TURBOMOLE quantum chemistry package.}
\cvitem{2}{Used FORTRAN and high performance computing techniques to implement quantum chemistry methods within TURBOMOLE}
\cvitem{3}{Collaborated with experimental chemists in the fields of inorganic lanthanide chemistry, surface chemistry and organic chemistry
to apply the code that I developed to study chemical phenomena computationally.}
%\cvlistdoubleitem{Protein dynamics}{Drug discovery}
%\cvlistdoubleitem{Density Functional Theory (DFT)}{QM/MM simulations}
%\cvlistdoubleitem{Time-dependent DFT (TDDFT)}{Random phase approximation (RPA)}

\section{Relevant Publications}
\cvitem{1}{\underline{S. G. Balasubramani}, D. Singh and R. S. Swathi, Noble gas encapsulation into carbon nanotubes: Predictions from analytical model and DFT studies, \textit{J. Chem. Phys.} \textbf{141}, 184304 (2014)}
\cvitem{2}{G. P. Chen, V. K. Voora, M. M. Agee, \underline{S. G. Balasubramani}, and F. Furche,
Random Phase Approximation Models, \textit{Annu. Rev. Phys. Chem.}, \textbf{68}, 445 (2017)}
%\cvitem{3}{A. J. Ryan, L. E. Darago, \underline{S. G. Balasubramani}, G. P. Chen, J. W. Ziller, F. Furche, J. R. Long, W. J. Evans, Synthesis, Structure, and Magnetism of Tris(amide)
%[Ln\{N(SiMe$_3$)$_2$\}$_3$]$^{1-}$ Complexes of the Non-traditional +2
%Lanthanide Ions, \textit{Chem. Eur. J.}, \textbf{24}, 7702 (2018)}
%\cvitem{4}{V. K. Voora, \underline{S. G. Balasubramani}, and F. Furche, Variational generalized Kohn-Sham approach combining the random-phase-approximation and Green's-function methods,
%\textit{Phys. Rev. A} \textbf{99}, 012518 (2019)}
\cvitem{3}{\underline{S. G. Balasubramani}, et al., TURBOMOLE: Modular program suite for ab initio quantum-chemical and condensed-matter simulations,
\textit{J. Chem. Phys.} \textbf{152}, 184107 (2020)}
%\cvitem{6}{A. J. Ryan, \underline{S. G. Balasubramani}, et al., Formation of the End-on Bound Lanthanide Dinitrogen Complexes [(R$_2$N)$_3$Ln-N-N-Ln(NR$_2$)$_3$]$^{2-}$ from Divalent [(R$_2$N)$_3$Ln]$^{1-}$ Salts (R = SiMe$_3$),
%\textit{J. Chem. Phys.} \textbf{152}, 184107 (2020)}

%\section{Masters Thesis}
%\cvitem{Title}{\emph{Theoretical studies of the encapsulation of noble gas atoms inside carbon nanotubes}}
%\cvitem{Supervisor}{\emph{Dr. R. S. Swathi}}
%\cvitem{2013-2014}{IISER-TVM}

%\section{PhD Thesis}
%\cvitem{Title}{Analytical molecular properties within the RPA}
%\cvitem{Supervisor}{\emph{Prof. Filipp Furche}}
%\cvitem{2014-Present}{University of California, Irvine}
%\cvitem{Description}{Development of a generalized Kohn-Sham RPA (GKS-RPA) method.
%Development and implementation of analytical second order 
%properties such as polarizabilities and excitation energies within the GKS-RPA.
%Implementation and testing of ionization potentials (IPs) and electron affinities (EAs)
%of molecules within GKS-RPA and inclusion of
%conductor like screening model (COSMO) to study the core-IPs of solvated molecules.  
%}

%----------------------------------------------------------------------------------------
%	WORK EXPERIENCE SECTION
%----------------------------------------------------------------------------------------

%\section{Current projects}

%\subsection{Research}

%\cvitem{}{Graduate research}
%\cvitem{Supervisor}{\textbf{Prof. Filipp Furche}}
%\cvitem{}{\emph{University of California, Irvine}}
%\cvitem{Duration}{March 2015--present}
%\cvitem{Title}{\textbf{Time-dependent random phase approximation}}

%\cvitem{}{Graduate research}
%\cvitem{Supervisors}{\textbf{Prof. Filipp Furche} and \textbf{Dr. Vamsee Voora}}
%\cvitem{Co-supervisor}{\textbf{Dr. Vamsee Voora}}
%\cvitem{}{\emph{University of California, Irvine}}
%\cvitem{Duration}{March 2015--present}
%\cvitem{Title}{\textbf{Orbital optimized random phase approximation}}


%\cvitem{}{Summer research}
%\cvitem{Supervisor}{\textbf{Prof. K. L. Sebastian}}
%\cvitem{}{\emph{Indian Institute of Science, Bangalore}}
%\cvitem{Duration}{May 2013--July 2013}
%\cvitem{Title}{\textbf{Plasmon hybridization model for spherical nanoparticles}}
%\cvitem{Description}{We studied the plasmon hybridization model for spherical nanoparticles and applied it to understand the plasmon energies and couplings in various geometries of spherical nanoparticle clusters.\newline}


%\cvitem{}{Semester long Project}
%\cvitem{Supervisor}{\textbf{Dr. Madhu Thalakulam}}
%\cvitem{}{\emph{Indian Institute of Science Education and Research, Thiruvananthapuram}}
%\cvitem{Duration}{January 2013--April 2013}
%\cvitem{Title}{\textbf{Design and Study of On-chip impedance transformers for RF reflectometry measurements.}}
%\cvitem{Description}{ Wrote simple Matlab codes to simulate various characteristics of RF circuits. Circuit elements such as inductor, capacitor and resistor were optimized to obtain maximum gain out of the RF circuit.\newline}

%\cvitem{}{Project}
%\cvitem{Supervisor}{\textbf{Dr. R. S. Swathi}}
%\cvitem{}{\emph{Indian Institute of Science Education and Research, Thiruvananthapuram}}
%\cvitem{Duration}{November 2012--February 2013}
%\cvitem{Title}{\textbf{Molecular electronics}}
%\cvitem{Description}{ Learned the theory of electronic transport across single molecules using the non-equilibrium Green's functions method and simulated the conductance across conjugated carbon rings for various orientations of the rings between two electrodes.\newline}

%\cvitem{}{Summer research}
%\cvitem{Supervisor}{\textbf{Dr. R. S. Swathi}}
%\cvitem{}{\emph{Indian Institute of Science Education and Research, Thiruvananthapuram}}
%\cvitem{Duration}{May 2012--July 2012}
%\cvitem{Title}{\textbf{The origin of van der Waals interaction}}
%\cvitem{Description}{Studied the quantum mechanical origin of van der Waals interaction between atoms using the second order perturbation theory.\newline}

%\cvitem{}{Summer research}
%\cvitem{Supervisor}{\textbf{Dr. R. S. Swathi}}
%\cvitem{}{\emph{Indian Institute of Science Education and Research, Thiruvananthapuram}}
%\cvitem{Duration}{May 2011--July 2011}
%\cvitem{Title}{\textbf{Theoretical investigations of plasmons in spherical gold nanoparticles.}}
%\cvitem{Description}{Studied the classical theory of plasmons. Used the Finite difference time domain (FDTD) method implemented in a software called Lumerical solutions to solve the Maxwell's equations and find the absorption and extinction cross-sections as well as the electric field enhancement around spherical gold nanoparticles of various sizes.\newline}

%\cvitem{}{Summer research}
%\cvitem{Supervisor}{\textbf{Dr. S. Anandan}}
%\cvitem{}{\emph{National Institute of Technology, Trichy}}
%\cvitem{Duration}{May 2010}
%\cvitem{Title}{\textbf{Fabrication and studies of dye sensitized solar cells}}
%\cvitem{Description}{Studied about the dye sensitized solar cells (DSSC). Learned experimental techniques for fabricating the same. Understood how to measure the efficiency of the DSSC.\newline}

%----------------------------------------------------------------------------------------
%	AWARDS SECTION
%----------------------------------------------------------------------------------------



%\section{References}
%\cvitem{1}{Prof. Filipp Furche, Department of Chemistry, UC Irvine, email: filipp.furche@uci.edu}
%\cvitem{2}{Prof. Kieron Burke, Department of Chemistry and Physics, UC Irvine,email: kieron@uci.edu}
%\cvitem{3}{Prof. Ioan Andricioaei, Department of Chemistry, UC Irvine,email: andricio@uci.edu}
%\cvitem{4}{Prof. Steven Schwartz, Chemistry and Biochemistry, University of Arizona,email: sschwartz@email.arizona.edu}
%----------------------------------------------------------------------------------------
%	CONFERENCES SECTION
%----------------------------------------------------------------------------------------
%\section{Conferences}
%\cventry{2016 February}{Computers in chemistry, 251st American Chemical Society National Meeting and Exposition}{San Diego}{Poster presentation}{Non-covalent interactions %using orbital optimized random phase approximation (oRPA)}{}
%\cventry{2016 June}{SoCal TheoChem Symposium}{UC San Diego}{Poster presentation}{Analytical second order properties within the RPA}{}
%\cventry{2017 May}{2$^{nd}$ annual SoCal TheoChem Symposium}{UC Irvine}{Poster presentation}{}{}
%\cventry{2017 July}{American Conference on Theoretical Chemistry (ACTC)}{Boston}{Poster presentation}{Second order molecular properties from the random phase approximation}{}

%----------------------------------------------------------------------------------------
%	TEACHING SECTION
%----------------------------------------------------------------------------------------

%\section{Teaching experience as a teaching assistant (TA)}
%\cvitem{Fall 2014}{Chem 1LD general chemistry laboratory}
%\cvitem{Winter 2015}{Chem 1A general chemistry lecture discussions}
%\cvitem{Spring 2015}{Chem 1LC general chemistry laboratory}
%\cvitem{Winter 2016}{Chem 1LE general chemistry laboratory for engineers}
%\cvitem{Spring 2016}{Chem 231 computational chemistry laboratory}
%\cvitem{Winter 2017}{Chem 252 density functional theory}
%\cvitem{Winter 2018}{Chem 150L computational chemistry laboratory}
\section{Skills}
\cvlistdoubleitem{Python}{FORTRAN}
\cvlistdoubleitem{Shell scripting}{LINUX}
\cvlistdoubleitem{TURBOMOLE}{CHARMM}
\cvlistdoubleitem{Git}{VMD}

\section{Honors and Awards}
\cvitem{2009-2014}{INSPIRE fellowship, Indian Institue of Science Education and Research, Trivandrum}


%----------------------------------------------------------------------------------------
%	INTERESTS SECTION
%----------------------------------------------------------------------------------------

%\section{Interests}

%\renewcommand{\listitemsymbol}{-~} % Changes the symbol used for lists

%\cvlistdoubleitem{Writing poems and stories}{Football}
%\cvlistdoubleitem{Cooking}{Reading books}
%\cvlistitem{Computer gaming}

%----------------------------------------------------------------------------------------
%	COVER LETTER
%----------------------------------------------------------------------------------------

% To remove the cover letter, comment out this entire block



\end{document}
